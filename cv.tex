\documentclass[10pt, a4paper]{article}

\usepackage[scaled]{helvet}
\renewcommand\familydefault{\sfdefault} 
\usepackage[T1]{fontenc}

% \usepackage{pagecolor}
% \usepackage{polyglossia}
% \usepackage{xltxtra}
\usepackage{geometry}

\geometry{a4paper, left=35mm, right=35mm, top=35mm, bottom=17mm}

% Do not indent paragraphs
\setlength\parindent{0in}

% Enable multicolumns
\usepackage{multicol}
\setlength{\columnsep}{-3.5cm}

% MARGIN NOTES
%--------------------------------
\usepackage{marginnote}
\newcommand{\note}[1]{\marginnote{\small #1}}
\renewcommand*{\raggedleftmarginnote}{}
\setlength{\marginparsep}{7pt}
\reversemarginpar

% HEADINGS
%--------------------------------
\usepackage{sectsty} 
\usepackage[normalem]{ulem} 
%\sectionfont{\rmfamily\mdseries} 
%\subsectionfont{\rmfamily\mdseries\scshape\normalsize} 
%\subsubsectionfont{\rmfamily\bfseries\upshape\normalsize} 

% PDF SETUP
%--------------------------------
\usepackage{hyperref}
\hypersetup
{
  pdfauthor={Samuel V. Hulse},
  pdfsubject={Samuel V. Hulse's CV},
  pdftitle={Samuel V. Hulse's CV},
  colorlinks, breaklinks, xetex, bookmarks,
  filecolor=black,
  urlcolor=[rgb]{0.117,0.682,0.858},
  linkcolor=[rgb]{0.117,0.682,0.858},
  linkcolor=[rgb]{0.117,0.682,0.858},
  citecolor=[rgb]{0.117,0.682,0.858}
}

\begin{document}

\definecolor{light-gray}{rgb}{0.935, 0.924, 0.908} 
\pagecolor{light-gray}

{\LARGE Samuel V. Hulse}\\[.4cm]
{\large Postdoctral Associate\\
Theoretical Evolutionary Biologist\\[0.5cm]}

\begin{multicols}{2}
  \textbf{Address:}\\
  1350 Shepherd St NW\\
  Washington, DC 20011\\[0.5cm]
  \columnbreak

  \textbf{Phone:} (443) 527-5710)\\
  \textbf{Email:} \href{mailto:$email$}{shulse@umd.edu}\\
  \textbf{GitHub:} svhulse
\end{multicols}

\section*{Professional Interests}
For my doctoral work, I focused on expanding the domain of the sensory drive model beyond peripheral
sensory processing, to explain the evolution of complex visual displays. In the Bruns Lab, I am working to
develop theoretical models for how evolutionary feedbacks influence disease resistance in plant models.
My understanding of the field has been greatly informed by my passion for mathematics, and I am
captivated by how evolutionary theory can be made more rigorous through interdisciplinary approaches.\\

\section*{Professional Experience}
\note{2021 - Curr.} \textbf{Postdoctoral Associate}\\
\emph{University of Maryland College Park}, Baltimore, MD\\
\emph{Supervisor:} Dr. Emily Bruns\\

\section*{Education}
\note{2021} \textbf{Ph.D., Biological Sciences}\\
\emph{University of Maryland Baltimore County}, Baltimore, MD\\
\emph{Dissertation:} The Evolution of Visual Patterning in North American Freshwater Fishes\\
\emph{Supervisor:} Dr. Tamra Mendelson\\[.2cm]

\note{2021} \textbf{M.S., Applied Mathematics}\\
\emph{University of Maryland Baltimore County}, Baltimore, MD\\[.2cm]

\note{2012}\textbf{B.S., Environmental Science}\\
\emph{Juniata College}, Huntingdon, PA\\



\section*{Publications}
\subsection*{\emph{Peer-Reviewed Publications}}
\note{2023}\textbf{Hulse, S.V.,} Antonovics, J., Hood, M.E., and Bruns, E.L.
Host-pathogen coevolution promotes the evolution of general, broad-spectrum resistance and reduces foreign pathogen spillover risk.
\emph{Accepted, Evolution Letters}.\\[.2cm]

\note{2023}\textbf{Hulse, S.V.,} Antonovics, J., Hood, M.E., and Bruns, E.L.
Specific resistance prevents the evolution of general resistance and facilitates disease emergence.
\emph{Journal of Evolutionary Biology} 36: 753-763.\\[.2cm]

\note{2022}\textbf{Hulse, S.V.,} Renoult, J.P., and Mendelson, T.C.
Using deep neural networks to model similarity between visual patterns: Application to fish sexual signals.
\emph{Ecological Informatics} 67: 101486.\\[.2cm]

\note{2020}\textbf{Hulse, S.V.,} Renoult, J.P., and Mendelson, T.C.
Sexual signaling pattern correlates with habitat pattern in visually ornamented fishes.
\emph{Nature Communications} 11: 2561.\\

\subsection*{\emph{Dissertation}}
\note{2021}\textbf{Hulse, S.V.}
The Evolution of Visual Patterning in North American Freshwater Fishes. \\[.2cm]


\section*{Conferences and Presentations}
\subsection*{\emph{Invited Talks}}
\note{2023} \textbf{Hulse, S.V.} 
The evolution and maintenance of host genetic diversity for pathogen resistance.
Mathematical Biology Seminar,
University of Maryland College Park.\\[0.2cm]

\note{2022} \textbf{Hulse, S.V.}
Applications of Deep Learning to Fish Behavioral Patterns.
Machine Learners Group Seminar,
Scripps Institution of Oceanography.\\[0.2cm]

\note{2019} \textbf{Hulse, S.V.}
Understanding the signals animals send each other.
High School Assembly Presentation,
The Park School of Baltimore.\\

\subsection*{\emph{Contributed Talks}}
\note{2023} \textbf{Hulse, S.V.}
A theoretical model for the shape of evolutionary tradeoffs.
Southeastern Population Ecology and Evolutionary Genetics, Pembroke, VA.\\[0.2cm]

\note{2023} \textbf{Hulse, S.V.}
The role of coevolution in mantaining host resistance structures.
Evolution, Albuquerque, NM.\\[0.2cm]

\note{2023} \textbf{Hulse, S.V.}
Does host-pathogen coevolution increase the risk of foreign pathogen invasion?
Ecology and Evolution of Infectious Diseases, State College, PA.\\[0.2cm]

\note{2021} \textbf{Hulse, S.V.}
Visual statistsics of habitat predict spatial aspect of visual signals.
University of Maryland Behavior, Ecology, Evolution, and Systematics Department Retreat, Thurmont, MD.\\[0.2cm]

\note{2019} Mendelson, T.C., \textbf{Hulse, S.V.,} Renoult, J.P.
Complex nuptial patterns of fish species mimic the spatial statistics of their habitat. 
Annual meeting of the Animal Behavior Society, Chicago, IL.\\[0.2cm]

\note{2018} \textbf{Hulse, S.V.}
The Efficient Coding Hypothesis and Signal Design.
UMBC Biological Sciences Departmental Seminar, Baltimore, MD.\\[0.2cm]

\note{2018} \textbf{Hulse, S.V.,} and Mendelson, T.C.
The efficient coding hypothesis and signal design.
Annual meeting of the Society for Integrative and Comparative Biology, San Francisco, CA.\\[0.2cm]

\note{2017} \textbf{Hulse, S.V.,} and Mendelson, T.C.
The efficient coding hypothesis and signal design.
Spotlight Talk, Evolution, Portland, OR.\\

\subsection*{\emph{Posters}}
\note{2022} \textbf{Hulse, S.V.,} and Bruns. E.L.
Disease Resistance at the Whole Organism Level, The Joint Evolution of General and Specific Resistance.
Ecology and Evolution of Infectious Diseases, Atlanta GA.\\[0.2cm]

\note{2020} \textbf{Hulse, S.V.,} Mendelson, T.C., and Renoult, J.P.
The spatial statistics of sexual signals in fishes correspond to their habitat: extending sensory drive to signal design. 
NSF workshop: Biology through Information Communication Coding Theory, Alexandria, VA.\\[0.2cm]

\note{2018} \textbf{Hulse, S.V.,} Renoult, J.P., and Mendelson, T.C.
The Efficient Coding Hypothesis and the Evolution of Signal Design.
Evolution, Montpellier, France.\\[0.2cm]

\note{2017} \textbf{Hulse, S.V.,} and Mendelson, T.C.
The efficient coding hypothesis and signal design.
Annual meeting of the Society for Integrative and
Comparative Biology, New Orleans, LA.\\


\section*{Grants, Awards, and Fellowships}
\subsection*{Fellowships}
\note{2019} Millhauser Fellowship, The Park School of Baltimore (\$250)\\[0.2cm]
\note{2018} Chateaubriand Fellowship, The Embassy of France in the United States (\$4200)\\

\subsection*{Travel Awards}
\note{2020} NSF BIOtIC Workshop Student Support (Housing Support)\\[0.2cm]
\note{2018} SICB Charlotte Magnum Student Support (Housing Support)\\[0.2cm]
\note{2018} SICB Charlotte Magnum Student Support (Housing Support)\\[0.2cm]
\note{2018} Wilson Ornithological Society Travel Award (\$285)\\

\subsection*{Other Awards}
\note{2018} AAAS/Science Program for Excellence in Science (Full AAAS Membership benefits)\\[0.2cm]


\section*{Training}
\note{2022} University of Maryland Mentoring Workshops for Postdoctoral Fellows, College Park, MD.\\[0.2cm]
\note{2020} MIT Brains, Minds Machines Virtual Summer Course,
Woods Hole, MA.\\



\section*{Teaching Experience}
\note{2023} Developing Course: BSCI 338V: Introduction to Python for Life Sciences\\[0.2cm]
\note{2015-2021} Teaching Assistant, Comparative Vertebrate Physiology Lab\\[0.2cm]
\note{2016-2020} Teaching Assistant, Anatomy and Physiology II Lab\\[0.2cm]
\note{2018} Guest Lecturer, Sexual Selection\\[0.2cm]
\note{2017, 2018} Guest Lecturer, Animal Behavior\\



\section*{Mentoring}
\subsection*{Undergraduate Mentoring, University of Maryland College Park}
\note{2023} Molly Gans, Amherst University\\[0.2cm]
\note{2022} Daniel Fu, University of Maryland College Park\\

\section*{Academic Service}
\subsection*{Manuscript Peer Reviewed}
\note{2023} Biology Letters (Joint review with Dr. Emily Bruns)\\[0.2cm]
\note{2022} Evolutionary Ecology\\[0.2cm]
\note{2020} Behavioral Ecology\\
\subsection*{Misc. Service}
\note{2023 - Curr.} Founder and Organiser: UMD Mathematical Biology Journal Club\\[0.2cm]
\note{2023} SSE W. D. Hamilton Award Judge\\[0.2cm]
\note{2023} Maryland Day 2023 Outreach Volunteer\\[0.2cm]
\note{2016-2020} UMBC Department of Biological Science FUN Committee\\[0.2cm]
\note{2016-2017} UMBC Graduate Student Association Senator

\vspace{25pt}

\end{document}
