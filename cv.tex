\documentclass[10pt, a4paper]{article}

\usepackage[scaled]{helvet}
\renewcommand\familydefault{\sfdefault} 
\usepackage[T1]{fontenc}

% \usepackage{pagecolor}
% \usepackage{polyglossia}
% \usepackage{xltxtra}
\usepackage{geometry}

\geometry{a4paper, left=35mm, right=35mm, top=35mm, bottom=17mm}

% Do not indent paragraphs
\setlength\parindent{0in}

% Enable multicolumns
\usepackage{multicol}
\setlength{\columnsep}{-3.5cm}

% MARGIN NOTES
%--------------------------------
\usepackage{marginnote}
\newcommand{\note}[1]{\marginnote{\small #1}}
\renewcommand*{\raggedleftmarginnote}{}
\setlength{\marginparsep}{7pt}
\reversemarginpar

% HEADINGS
%--------------------------------
\usepackage{sectsty} 
\usepackage[normalem]{ulem} 
%\sectionfont{\rmfamily\mdseries} 
%\subsectionfont{\rmfamily\mdseries\scshape\normalsize} 
%\subsubsectionfont{\rmfamily\bfseries\upshape\normalsize} 

% PDF SETUP
%--------------------------------
\usepackage{hyperref}
\hypersetup
{
  pdfauthor={Samuel V. Hulse},
  pdfsubject={Samuel V. Hulse's CV},
  pdftitle={Samuel V. Hulse's CV},
  colorlinks, breaklinks, xetex, bookmarks,
  filecolor=black,
  urlcolor=[rgb]{0.117,0.682,0.858},
  linkcolor=[rgb]{0.117,0.682,0.858},
  linkcolor=[rgb]{0.117,0.682,0.858},
  citecolor=[rgb]{0.117,0.682,0.858}
}

\begin{document}

\definecolor{light-gray}{rgb}{0.935, 0.924, 0.908} 
%\pagecolor{light-gray}

{\LARGE Samuel V. Hulse}\\[.4cm]
{\large Postdoctoral Associate\\
Mathematical Evolutionary Biologist\\[0.5cm]}

\begin{multicols}{2}
  \textbf{Address:}\\
  1350 Shepherd St NW\\
  Washington, DC 20011\\[0.5cm]
  \columnbreak

  \textbf{Phone:} (443) 527-5710\\
  \textbf{Email:} \href{mailto:$email$}{shulse@umd.edu}\\
  \textbf{GitHub:} svhulse
\end{multicols}

\section*{Education}
\note{2021} \textbf{Ph.D., Biological Sciences}, University of Maryland Baltimore County\\
Supervisor: Dr. Tamra Mendelson\\

\note{2021} \textbf{M.S., Applied Mathematics}, University of Maryland Baltimore County\\

\note{2012}\textbf{B.S., Environmental Science}, Juniata College\\

\section*{Professional Appointments}
\note{2021 - Curr.} \textbf{Postdoctoral Associate}, University of Maryland College Park\\
Supervisor: Dr. Emily Bruns\\

\section*{Publications}
\subsection*{Peer-Reviewed Publications}
\note{2024}\textbf{Hulse, S.V.,} and Bruns, E.L.
The Emergence of Non-Linear Evolutionary Trade-offs and the Maintenance of Genetic Polymorphisms. 
\emph{Biology Letters} 20: 20240296. \\

\note{2023}\textbf{Hulse, S.V.,} Antonovics, J., Hood, M.E., and Bruns, E.L.
Host-pathogen coevolution promotes the evolution of general, broad-spectrum resistance and reduces foreign pathogen spillover risk.
\emph{Evolution Letters} 7: 467-477.\\

\note{2023}\textbf{Hulse, S.V.,} Antonovics, J., Hood, M.E., and Bruns, E.L.
Specific resistance prevents the evolution of general resistance and facilitates disease emergence.
\emph{Journal of Evolutionary Biology} 36: 753-763.\\

\note{2022}\textbf{Hulse, S.V.,} Renoult, J.P., and Mendelson, T.C.
Using deep neural networks to model similarity between visual patterns: Application to fish sexual signals.
\emph{Ecological Informatics} 67: 101486.\\

\note{2020}\textbf{Hulse, S.V.,} Renoult, J.P., and Mendelson, T.C.
Sexual signaling pattern correlates with habitat pattern in visually ornamented fishes.
\emph{Nature Communications} 11: 2561.\\

\subsection*{Dissertation}
\note{2021}\textbf{Hulse, S.V.}
The Evolution of Visual Patterning in North American Freshwater Fishes. \\

\section*{Conferences and Presentations}
\subsection*{Invited Talks}
\note{2023} \textbf{Hulse, S.V.}
Does host-pathogen coevolution increase the risk of foreign pathogen invasion?
Ecology and Evolution of Infectious Diseases, State College, PA.\\

\note{2023} \textbf{Hulse, S.V.} 
The evolution and maintenance of host genetic diversity for pathogen resistance.
Mathematical Biology Seminar,
University of Maryland College Park.\\

\note{2022} \textbf{Hulse, S.V.}
Applications of Deep Learning to Fish Behavioral Patterns.
Machine Learners Group Seminar,
Scripps Institution of Oceanography.\\

\subsection*{Contributed Talks}
\note{2024} \textbf{Hulse, S.V.}
Host-Pathogen Coevolution Makes or Breaks Transmission Modes.
Symposium Talk, Evolution, Montreal, QC.\\

\note{2023} \textbf{Hulse, S.V.}
A theoretical model for the shape of evolutionary trade-offs.
Southeastern Population Ecology and Evolutionary Genetics, Pembroke, VA.\\

\note{2023} \textbf{Hulse, S.V.}
The role of coevolution in maintaining host resistance structures.
Evolution, Albuquerque, NM.\\

\note{2021} \textbf{Hulse, S.V.}
Visual statistics of habitat predict spatial aspect of visual signals.
University of Maryland Behavior, Ecology, Evolution, and Systematics Department Retreat, Thurmont, MD.\\

\note{2018} \textbf{Hulse, S.V.}
The Efficient Coding Hypothesis and Signal Design.
UMBC Biological Sciences Departmental Seminar, Baltimore, MD.\\

\note{2018} \textbf{Hulse, S.V.,} and Mendelson, T.C.
The efficient coding hypothesis and signal design.
Annual meeting of the Society for Integrative and Comparative Biology, San Francisco, CA.\\

\note{2017} \textbf{Hulse, S.V.,} and Mendelson, T.C.
The efficient coding hypothesis and signal design.
Spotlight Talk, Evolution, Portland, OR.\\

\subsection*{Posters}
\note{2022} \textbf{Hulse, S.V.,} and Bruns. E.L.
Disease Resistance at the Whole Organism Level, The Joint Evolution of General and Specific Resistance.
Ecology and Evolution of Infectious Diseases, Atlanta GA.\\

\note{2020} \textbf{Hulse, S.V.,} Mendelson, T.C., and Renoult, J.P.
The spatial statistics of sexual signals in fishes correspond to their habitat: extending sensory drive to signal design. 
NSF workshop: Biology through Information Communication Coding Theory, Alexandria, VA.\\

\note{2018} \textbf{Hulse, S.V.,} Renoult, J.P., and Mendelson, T.C.
The Efficient Coding Hypothesis and the Evolution of Signal Design.
Evolution, Montpellier, France.\\

\note{2017} \textbf{Hulse, S.V.,} and Mendelson, T.C.
The efficient coding hypothesis and signal design.
Annual meeting of the Society for Integrative and
Comparative Biology, New Orleans, LA.\\

\subsection*{Outreach}
\note{2019} \textbf{Hulse, S.V.}
Understanding the signals animals send each other.
High School Assembly Presentation,
The Park School of Baltimore.\\

\section*{Grants, Awards, and Fellowships}
\subsection*{Fellowships}
\note{2018} Chateaubriand Fellowship, The Embassy of France in the United States (\$4200)\\

\subsection*{Travel Awards}
\note{2020} NSF BIOtIC Workshop Student Support (Housing and Travel Support)\\
\note{2018} SICB Charlotte Magnum Student Support (Housing Support)\\
\note{2018} SICB Charlotte Magnum Student Support (Housing Support)\\
\note{2018} Wilson Ornithological Society Travel Award (\$285)\\

\subsection*{Other Awards}
\note{2018} AAAS/Science Program for Excellence in Science (Full AAAS Membership benefits)\\


\section*{Training}
\note{2022} University of Maryland Mentoring Workshops for Postdoctoral Fellows, College Park, MD.\\
\note{2020} MIT Brains, Minds Machines Virtual Summer Course,
Woods Hole, MA.\\



\section*{Teaching Experience}
\subsection*{\emph{Instructor of Record}}
\note{2024} Introduction to Python for Life Sciences.\\
\textit{Developed and taught an undergraduate course designed to introduce biologists to the python programing language. 
Through this course, I taught my students the basics of the python language, as well as numerical simulations and data analysis using numpy, pandas, and matplotlib.
Students engaged with real biological data, such as the Iris dataset, and implemented basic epidemiological models.} \\

\subsection*{\emph{Teaching Assistant Roles and Guest Lectures}}
\note{2023} Guest Lecturer, Principles of Ecology and Evolution\\
\note{2015-2021} Teaching Assistant, Comparative Vertebrate Physiology Lab\\
\note{2016-2020} Teaching Assistant, Anatomy and Physiology II Lab\\
\note{2018} Guest Lecturer, Advanced Topics in Ecology and Evolution: Sexual Selection\\
\note{2017-2018} Guest Lecturer, Animal Behavior\\



\section*{Mentoring}
\subsection*{Undergraduate Mentoring}
\note{2024} Bhargav Srinivasan, Undergraduate Student, University of Maryland College Park\\
\textit{I have been working with Bhargav to develop a model to predict whether it is advantageous for infected plants to flower prior to susceptible plants.}

\note{2023} Molly Gans, Visting Undergraduate Student from Amherst University\\
\textit{I worked with Molly to develop a modeling component of her senior thesis, using linear systems of differential equations to model the infection process.}

\note{2022} Daniel Fu, Undergraduate Student, University of Maryland College Park\\
\textit{I supervised Daniel to develop a model for predicting when evolution would favor sterility virulence versus mortality virulence. }

\section*{Academic Service}
\subsection*{Peer Reviewing}
\note{2024} Evolutionary Applications\\
\note{2024} Ecology and Evolution\\
\note{2024} Grant Reviewer: Deutsche Forschungsgemeinschaft\\
\note{2023} Biology Letters (Joint review with Dr. Emily Bruns)\\
\note{2022} Evolutionary Ecology\\
\note{2020} Behavioral Ecology\\
\subsection*{Other Service}
\note{2025} Committee Member: Biology Department Committee for Maryland Day Outreach\\
\note{2024 - Curr} Organizing Committee Member: Quantitative Ecology and Evolution Dynamics Group\\
\note{2023 - 2024} Founder and Organizer: UMD Mathematical Biology Journal Club\\
\note{2023} Poster Judge, Southeastern Population Ecological and Evolutionary Genetics 2023\\
\note{2023} SSE W. D. Hamilton Award Judge\\
\note{2023} Maryland Day 2023 Outreach Volunteer\\
\note{2016-2020} UMBC Department of Biological Science FUN Committee\\
\note{2016-2017} UMBC Graduate Student Association Senator

\vspace{25pt}



%\section*{References}
%\textbf{Dr. Emily Bruns}\\
%Assistant Professor in Biology\\
%University of Maryland College Park\\
%Phone: (612) 360-1901\\
%Email: \href{mailto:$email$}{ebruns@umd.edu}\\

%Dr. Bruns is my current postdoc advisor.\\

%\textbf{Dr. Tamra Mendelson}\\
%Professor in Biological Sciences\\
%University of Maryland Baltimore County\\
%Phone: (301) 404-8651\\
%Email: \href{mailto:$email$}{tamram@umbc.edu}\\

%Dr. Mendelson was my PhD advisor.\\

%\textbf{Dr. Sarah Leupen}\\
%Principal Lecturer in Biological Sciences\\
%University of Maryland Baltimore County\\
%Phone: (410) 564-6945\\
%Email: \href{mailto:$email$}{leupen@umbc.edu}\\

%I was a teaching assistant for Dr. Leupen's comparative vertebrate physiology lab thoughout graduate school.

%\vspace{25pt}
\end{document}
